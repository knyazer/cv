\documentclass[11pt, a4paper]{article}
\usepackage[utf8]{inputenc}
\usepackage{geometry}
\usepackage{enumitem}
\usepackage{hyperref}
\usepackage{xcolor}
\usepackage{fontawesome5}

\geometry{left=2.5cm, right=2.5cm, top=1.5cm, bottom=1.5cm}
\pagenumbering{gobble}
\definecolor{primaryColor}{RGB}{0, 102, 204}

\setlength{\parindent}{0pt}
\setlength{\parskip}{1.5ex}

\newcommand{\entrytitle}[3]{
    \noindent\textbf{#1}, #2\hfill#3
}

\newlist{details}{itemize}{1}
\setlist[details]{
    label=\textbullet,
    leftmargin=1.5em,
    topsep=0.2ex,
    partopsep=-0pt,
    itemsep=0.2ex,
    parsep=0pt
}

\hypersetup{
    colorlinks=true,
    linkcolor=primaryColor,
    filecolor=primaryColor,
    urlcolor=primaryColor,
    citecolor=primaryColor,
}

\setlist[itemize]{noitemsep, topsep=0pt, leftmargin=*}

\newcommand{\sectiontitle}[1]{\vspace{3mm}{\color{primaryColor}\noindent\textbf{#1}}\vspace{1.5mm}\hrule\vspace{2mm}}
\newcommand{\experience}[4]{\textbf{#1}, #2 \hfill \vspace{-4mm}#3\\#4\vspace{3mm}}

\begin{document}

\begin{center}
    \LARGE{\textbf{\color{primaryColor}Roman Knyazhitskiy}}\\
    \vspace{2mm}
    \normalsize{
        \faEnvelope~\href{mailto:cv@knyaz.tech}{cv@knyaz.tech} \quad|\quad
        \faGlobe~\href{https://knyaz.tech}{knyaz.tech} \quad|\quad
        \faGithub~\href{https://github.com/knyazer}{knyazer}
    }
\end{center}

\sectiontitle{Summary}
\noindent
Master's student with 2 years of work experience in robotics and AI. Designed control systems for robots with complex physical interactions, utilizing both model-based (MPC) and data-driven (RL) control techniques. Experienced in custom physics simulators in JAX, contributed to various OSS libraries, and conducted some machine learning reasearch. Familiar with ROS.

\vspace{-2mm}


\sectiontitle{Education}

\entrytitle{MPhil in MLMI (Machine Learning)}{University of Cambridge}{10/2025 - 09/2026}

\entrytitle{BSc Computer Science and Engineering}{TU Delft}{09/2022 - 07/2025}
\begin{details}
    \item GPA: 8.7/10. Distinction (Cum Laude, top 5\%) + Honours.
\end{details}

\vspace{-2mm}


\sectiontitle{Work Experience}

\entrytitle{Machine Learning Engineer}{Delft Mercurians}{05/2023 - 09/2025}
\begin{details}
    \item Started as an individual contributor, then grew my own team to 5 people. Then abandoned the team to become an individual contributor in Software department.
    \item Designed a Model Predictive Control (MPC) system from ground up, vastly improving control performance compared to bang-bang.
    \item Co-authored a continuous-time differentiable physics simulator in JAX for training control policies via reinforcement learning.
    \item Worked on integration Python-based ML models with multithreaded Rust codebase.
\end{details}

\entrytitle{Research Associate}{TU Delft}{03/2023 - 08/2025}
\begin{details}
    \item Researched applications of transformer-based models (Prior-Data Fitted Networks) for meta-learning.
    \item Investigated LLM applications in software engineering and contributed to research on diffusion models.
\end{details}

\entrytitle{Applied Machine Learning Intern}{Central Robotics Institute}{06/2021 - 07/2021}
\begin{details}
    \item Developed computer vision algorithms for a robotic drawing application, including image segmentation and path optimization.
\end{details}

\vspace{-2mm}

\sectiontitle{Publications}

\vspace{-5mm}

\renewcommand{\refname}{}
\renewcommand{\refname}{\vspace{-\baselineskip}}

\bibliographystyle{unsrt}
\nocite{*}
\bibliography{pub}

\vspace{-3mm}

\sectiontitle{Selected Projects}

\entrytitle{Bootstrapped DQN Scaling Laws}{JAX, Deep RL Research, HPC}{2025}
\begin{details}
    \item Conducted a large-scale deep reinforcement learning study (>40,000 experiments), discovering a novel scaling law for bootstrapped exploration methods.
\end{details}

\entrytitle{Stack-Associated Beam Tracing}{C++20, 3D Graphics, Computational Geometry}{2022}
\begin{details}
    \item Implemented a 3D rendering engine in C++20 based on sparse voxel octree traversal for efficient beam queries.
\end{details}

\entrytitle{Lyapunov-Stabilized Continuous Control}{JAX, Deep RL, Robotics Simulation}{2024}
\begin{details}
    \item Developed an RL framework for continuous control (locomotion) using truncated backpropagation stabilized by Lyapunov factors, enabling stable gradient flow in long-horizon tasks within the Brax physics engine.
\end{details}

\entrytitle{Speculative decoding and multi-token prediction}{JAX, Equinox, Deep Learning}{2023}
\begin{details}
    \item Implemented a GPT-style autoregressive transformer from scratch in JAX/Equinox, architected for distributed training and featuring speculative decoding and multi-token prediction.
\end{details}

\vspace{-2mm}

\sectiontitle{Honours and Awards}
\begin{itemize}
    \item \textbf{Best Software Solution}, RoboCup World Championships, Sydney (2019).
    \item \textbf{1st Place}, RoboCup Junior National Competitions (2017, 2018, 2019).
    \item \textbf{2nd Place \& Special Prize}, Epoch AI Hackathon (2024).
    \item \textbf{1st Place}, Bunq Hackathon 6 (2025).
    \item \textbf{Silver Medal}, AIIJC International AI Competition for Juniors -- Sign language recognition application.
\end{itemize}

\vspace{-2mm}

\sectiontitle{Open Source Contributions}
\begin{itemize}
    \item Improved \textbf{libccd (collision detection library in C++)} by fixing a critical infinite loop bug.
    \item Enhanced functionalities in the JAX ecosystem libraries \textbf{jaxtyping} and \textbf{Equinox}.
    \item Contributed to \textbf{Gymnax}, a widely used collection of JAX-based RL environments.
\end{itemize}

\end{document}
