\documentclass[11pt, a4paper]{article}
\usepackage[utf8]{inputenc}
\usepackage{geometry}
\usepackage{enumitem}
\usepackage{hyperref}
\usepackage{xcolor}
\usepackage{fontawesome5}

\geometry{left=2.5cm, right=2.5cm, top=1.5cm, bottom=1.5cm}
\pagenumbering{gobble}
\definecolor{primaryColor}{RGB}{0, 102, 204}

\setlength{\parindent}{0pt}
\setlength{\parskip}{1.5ex}

\newcommand{\entrytitle}[3]{
    \noindent\textbf{#1}, #2\hfill#3
}

\newlist{details}{itemize}{1}
\setlist[details]{
    label=\textbullet,
    leftmargin=1.5em,
    topsep=0.2ex,
    partopsep=-0pt,
    itemsep=0.2ex,
    parsep=0pt
}

\hypersetup{
    colorlinks=true,
    linkcolor=primaryColor,
    filecolor=primaryColor,
    urlcolor=primaryColor,
    citecolor=primaryColor,
}

\setlist[itemize]{noitemsep, topsep=0pt, leftmargin=*}

\newcommand{\sectiontitle}[1]{\vspace{3mm}{\color{primaryColor}\noindent\textbf{#1}}\vspace{1.5mm}\hrule\vspace{2mm}}
\newcommand{\experience}[4]{\textbf{#1}, #2 \hfill \vspace{-4mm}#3\\#4\vspace{3mm}}

\begin{document}

\begin{center}
    \LARGE{\textbf{\color{primaryColor}Roman Knyazhitskiy}}\\
    \vspace{2mm}
    \normalsize{
        \faEnvelope~\href{mailto:cv@knyaz.tech}{cv@knyaz.tech} \quad|\quad
        \faGlobe~\href{https://knyaz.tech}{knyaz.tech} \quad|\quad
        \faGithub~\href{https://github.com/knyazer}{knyazer}
    }
\end{center}

\sectiontitle{Summary}
\noindent
Machine Learning Engineer with expertise in PyTorch/JAX and deep learning fundamentals, specializing in model optimization and scaling. Experience developing efficient ML systems for robotics and reinforcement learning, with strong Python/C++ skills and contributions to open-source ML libraries. Research background in AutoML, Meta Learning, and LLM applications with publications in top-tier venues.

\sectiontitle{Education}

\entrytitle{MPhil in MLMI (Machine Learning)}{University of Cambridge}{10/2025 - 09/2026}

\entrytitle{BSc Computer Science and Engineering}{TU Delft}{09/2022 - 07/2025}
\begin{details}
    \item GPA: 8.7/10. Cum Laude + Honours.
\end{details}

\sectiontitle{Work Experience}

\entrytitle{Research Associate}{TU Delft}{03/2023 - 08/2025}
\begin{details}
    \item Researched LLM applications including vulnerability detection and code generation
    \item Investigated Prior-Data Fitted Networks (PFNs) and MCMC acceleration methods
    \item Delivered talks on optimization methods including Adam's $L_\infty$ norm properties
    \item Participated in academic discussions on diffusion models and Meta Learning
\end{details}

\entrytitle{Machine Learning Engineer}{Delft Mercurians}{05/2023 - 09/2025}
\begin{details}
    \item Designed and implemented Model Predictive Control system for real-time trajectory optimization
    \item Built JAX+Equinox differentiable simulator tailored for robotic systems
    \item Integrated Python-based models into Rust codebase with communication protocols
    \item Contributed to open-source JAX-based ML libraries during development
\end{details}

\sectiontitle{Publications}

\vspace{-5mm}

\renewcommand{\refname}{}
\renewcommand{\refname}{\vspace{-\baselineskip}}

\bibliographystyle{unsrt}
\nocite{*}
\bibliography{pub}

\sectiontitle{Selected Projects}

\entrytitle{Bootstrapped DQN Scaling Laws}{JAX, Deep RL Research}{2025}
\begin{details}
    \item Conducted large-scale empirical study (>40,000 configurations) of ensemble-based exploration methods
    \item Implemented Bootstrapped DQN and Randomized-Prior BDQN in JAX with multi-GPU support
    \item Discovered scaling laws governing convergence, showing RP-BDQN solves ~1.5× harder tasks
    \item Optimized training with XLA compilation, bfloat16 precision, and automatic multi-GPU sharding
\end{details}

\entrytitle{Nano JAX GPT}{JAX, Equinox, Deep Learning}{2023}
\begin{details}
    \item Implemented high-performance GPT transformer with custom Flash Attention in JAX
    \item Supported distributed training across multiple accelerators using PositionalSharding
    \item Explored optimization techniques including RoPE embeddings, RMSNorm, and early-exit decoding
    \item Built training infrastructure with AdamW optimizer, gradient clipping, and checkpoint management
\end{details}

\entrytitle{IEFT-PFN: Inference Efficient Freeze-Thaw Prior Fitted Networks}{JAX, Equinox, Deep Learning}{2025}
\begin{details}
    \item Developed transformer-based system for hyperparameter optimization using in-context learning
    \item Implemented three hyperparameter weighting schemes for multi-curve aggregation
    \item Achieved 15-20\% improvement over baseline through learned attention weighting
    \item Optimized training with gradient accumulation and mixed precision handling
\end{details}

\entrytitle{Weather forecasting and option trading system}{Python, JAX/Equinox, Machine Learning}{2025}
\begin{details}
    \item Developed end-to-end algorithmic trading system integrating probabilistic ML models
    \item Implemented Continuous Normalizing Flow model using neural ODEs for uncertainty-calibrated predictions
    \item Designed frontier-based decision framework using epistemic uncertainty bounds
    \item Leveraged JAX's XLA compilation and hardware acceleration for efficient training
\end{details}

\sectiontitle{Open Source Contributions}
\begin{itemize}
    \item Enhanced functionalities in \href{https://github.com/google/jaxtyping}{jaxtyping} and \href{https://github.com/patrick-kidger/equinox}{Equinox}
    \item Contributed to \href{https://github.com/RobertTLange/gymnax}{Gymnax}, a JAX RL environments collection
    \item Improved libccd (collision detection library) fixing critical corner-case infinite loop
\end{itemize}

\sectiontitle{Honours and Awards}
\begin{itemize}
    \item \textbf{1st Place}, Bunq Hackathon 6 (2025) -- Team of 4 against 50+ teams, €30,000 prize
    \item \textbf{2nd Place \& Special Prize}, Epoch AI Hackathon (2024)
    \item \textbf{Silver Medal}, AIIJC International AI Competition for Juniors
\end{itemize}

\end{document}