\documentclass[11pt, a4paper]{article}
\usepackage[utf8]{inputenc}
\usepackage{geometry}
\usepackage{enumitem}
\usepackage{hyperref}
\usepackage{xcolor}
\usepackage{fontawesome5} % For icons like GitHub/LinkedIn

\geometry{left=2.5cm, right=2.5cm, top=1.5cm, bottom=1.5cm}
\pagenumbering{gobble}
\definecolor{primaryColor}{RGB}{0, 102, 204}

\setlength{\parindent}{0pt}
\setlength{\parskip}{1.5ex} % This controls the space BETWEEN entries. Adjust as needed.

% Command for a standard entry title line (left-aligned title, right-aligned date)
\newcommand{\entrytitle}[3]{%
    \noindent\textbf{#1}, #2\hfill#3%
}

% A custom list for details within an entry (e.g., GPA, bullet points)
% This is very compact and sits tightly under the title.
\newlist{details}{itemize}{1}
\setlist[details]{
    label=\textbullet,
    leftmargin=1.5em,
    topsep=0.2ex,
    partopsep=-0pt,
    itemsep=0.2ex,
    parsep=0pt
}

\hypersetup{
    colorlinks=true,
    linkcolor=primaryColor,
    filecolor=primaryColor,
    urlcolor=primaryColor,
    citecolor=primaryColor,
}

\setlist[itemize]{noitemsep, topsep=0pt, leftmargin=*}

\newcommand{\sectiontitle}[1]{\vspace{3mm}{\color{primaryColor}\noindent\textbf{#1}}\vspace{1.5mm}\hrule\vspace{2mm}}
\newcommand{\experience}[4]{\textbf{#1}, #2 \hfill \vspace{-4mm}#3\\#4\vspace{3mm}}

\begin{document}

\begin{center}
    \LARGE{\textbf{\color{primaryColor}Roman Knyazhitskiy}}\\
    \vspace{2mm}
    \normalsize{
        \faEnvelope~\href{mailto:cv@knyaz.tech}{cv@knyaz.tech} \quad|\quad
        \faGlobe~\href{https://knyaz.tech}{knyaz.tech} \quad|\quad
        \faGithub~\href{https://github.com/knyazer}{knyazer}
    }
\end{center}

\sectiontitle{Summary}
\noindent
MPhil student in Machine Learning at Cambridge with research experience in differentiable
simulation, computational finance and reinforcement learning. Developed methods improving stability in
BPTT, contributed to AutoML and meta-learning projects under Prof. Tom Viering, and
co-authored papers under submission. Also maintains open-source contributions to JAX-based ML libraries.

\vspace{-2mm}


\sectiontitle{Education}

\entrytitle{MPhil in MLMI (Machine Learning)}{University of Cambridge}{10/2025 - 09/2026}

\entrytitle{BSc Computer Science and Engineering}{TU Delft}{09/2022 - 07/2025}
\begin{details}
    \item Minor in Finance
    \item GPA: 8.7/10. Graduated with Honours
\end{details}

\vspace{-2mm}


\sectiontitle{Work Experience}

\entrytitle{Research Associate}{TU Delft}{03/2023 - 08/2025}
\begin{details}
    \item Under supervision of Professor Tom Viering worked on several AutoML and Meta Learning
    projects, including two yet to be published papers.
    \item Participated in departmental research meetings and academic discussions.
    \item Scaled experiments with HPC clusters, including a university-wide SLURM, Runpod and Google Cloud TPUs.
\end{details}

\entrytitle{Head of AI/Control}{Delft Mercurians (Student Robotics Team)}{05/2023 - 09/2025 (Part-Time)}
\begin{details}
    \item Led a team of 2-5 engineers in developing AI control systems for RoboCup competition.
    \item Designed and implemented a Model Predictive Control (MPC) system for real-time trajectory optimization.
    \item Worked closely with the higher management to ensure adequate integration and planning.
    \item Ensured AI solution robustness via runtime type checking, and a comprehensive test suite.
\end{details}

\vspace{-2mm}

\sectiontitle{Publications}

\vspace{-5mm}

\renewcommand{\refname}{}
\renewcommand{\refname}{\vspace{-\baselineskip}}

\bibliographystyle{unsrt}
\nocite{*}
\bibliography{pub} 

\vspace{-3mm}

\sectiontitle{Honours and Awards}
\begin{itemize}
    \item \textbf{1st Place}, Bunq Hackathon 6 (2025), team of 4 against 50+ teams with a prize of €30,000.
    \item \textbf{2nd Place \& Special Prize}, Epoch AI Hackathon (2024).
    \item \textbf{'Best Software Solution'} award, RoboCup World Championships, Sydney (2019).
    \item \textbf{1st Place}, RoboCup Junior National Competitions (2017, 2018, 2019).
\end{itemize}


\vspace{-2mm}

\sectiontitle{Open Source Contributions}
\begin{itemize}
    \item Enhanced functionalities in \href{https://github.com/google/jaxtyping}{jaxtyping} and \href{https://github.com/patrick-kidger/equinox}{Equinox}, resolving multiple issues and enabling IPython runtime type checking.
    \item Contributed to \href{https://github.com/RobertTLange/gymnax}{Gymnax}, a widely used JAX RL environments collection with 800+ stars on GitHub.
\end{itemize}


\vspace{-2mm}


\sectiontitle{Selected Projects}

\entrytitle{Marketmaking and pricing system for binary options}{Finance}{2025}
\begin{details}
    \item Designed and deployed a fully automated market-making bot on the Polymarket platform.
    \item Employed stochastic models to price binary options and developed a low-latency, extendable and multi-threaded execution system using Python.
\end{details}

\entrytitle{High-Frequency Weather Forecasting}{Hierarchical Bayesian Models, CNF}{2025}
\begin{details}
    \item Developed hybrid forecasting system combining continuous normalizing flows with an autoregressive model for well-calibrated forecasting.
\end{details}

\entrytitle{Lyapunov Discounting for BPTT Optimization}{JAX, Differentiable Simulation}{2024}
\begin{details}
    \item Novel stability improvement method for backpropagation through differentiable simulators, achieving significant gains over windowed BPTT, 5-fold increase in the peak performance achieved on Brax environments
\end{details}

\entrytitle{pytest-mut: High-Performance Mutation Testing}{Python, Parallel Computing}{2024}
\begin{details}
    \item Developed mutation testing library achieving 10-15x speedup over alternatives through novel partial parallelization strategies
\end{details}


\end{document}
