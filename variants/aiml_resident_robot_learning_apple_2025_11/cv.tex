\documentclass[11pt, a4paper]{article}
\usepackage[utf8]{inputenc}
\usepackage{geometry}
\usepackage{enumitem}
\usepackage{hyperref}
\usepackage{xcolor}
\usepackage{fontawesome5}

\geometry{left=2.5cm, right=2.5cm, top=1.5cm, bottom=1.5cm}
\pagenumbering{gobble}
\definecolor{primaryColor}{RGB}{0, 102, 204}

\setlength{\parindent}{0pt}
\setlength{\parskip}{1.5ex}

\newcommand{\entrytitle}[3]{
    \noindent\textbf{#1}, #2\hfill#3
}

\newlist{details}{itemize}{1}
\setlist[details]{
    label=\textbullet,
    leftmargin=1.5em,
    topsep=0.2ex,
    partopsep=-0pt,
    itemsep=0.2ex,
    parsep=0pt
}

\hypersetup{
    colorlinks=true,
    linkcolor=primaryColor,
    filecolor=primaryColor,
    urlcolor=primaryColor,
    citecolor=primaryColor,
}

\setlist[itemize]{noitemsep, topsep=0pt, leftmargin=*}

\newcommand{\sectiontitle}[1]{\vspace{3mm}{\color{primaryColor}\noindent\textbf{#1}}\vspace{1.5mm}\hrule\vspace{2mm}}
\newcommand{\experience}[4]{\textbf{#1}, #2 \hfill \vspace{-4mm}#3\\#4\vspace{3mm}}

\begin{document}

\begin{center}
    \LARGE{\textbf{\color{primaryColor}Roman Knyazhitskiy}}\\
    \vspace{2mm}
    \normalsize{
        \faEnvelope~\href{mailto:cv@knyaz.tech}{cv@knyaz.tech} \quad|\quad
        \faGlobe~\href{https://knyaz.tech}{knyaz.tech} \quad|\quad
        \faGithub~\href{https://github.com/knyazer}{knyazer}
    }
\end{center}

\sectiontitle{Summary}
\noindent
Master's student specializing in Robot Learning and Generative AI, with a proven track record in developing novel reinforcement learning algorithms and control systems for robotics. Led ML projects, from research and prototyping in JAX to building complete software systems in Python and C++. Eager to pursue research leading to publications in top-tier robotics conferences.

\vspace{-2mm}


\sectiontitle{Education}

\entrytitle{MPhil in MLMI (Machine Learning)}{University of Cambridge}{10/2025 - 09/2026}

\entrytitle{BSc Computer Science and Engineering}{TU Delft}{09/2022 - 07/2025}
\begin{details}
    \item GPA: 8.7/10. Distinction (Cum Laude, top 5\%) + Honours.
\end{details}

\vspace{-2mm}

\sectiontitle{Work Experience}

\entrytitle{Machine Learning Engineer}{Delft Mercurians RoboCup Team}{05/2023 - 09/2025}
\begin{details}
    \item Started out as an individual contributor, grew a team to 5 engineers in developing AI control systems for autonomous soccer robots.
    \item Designed and implemented a JAX-based differentiable physics simulator for training control models.
    \item Developed a Model Predictive Control (MPC) system for real-time trajectory optimization.
    \item Integrated Python-based ML models with a performance-critical Rust codebase.
\end{details}

\entrytitle{Research Associate}{TU Delft}{03/2023 - 08/2025}
\begin{details}
    \item Researched applications of Large Language Models (LLMs) in software engineering and investigated Prior-Data Fitted Networks (PFNs) for meta-learning.
    \item Participated in research discussions and delivered talks on diffusion models and advanced optimizers.
\end{details}

\entrytitle{Applied Machine Learning Intern}{Central Robotics Institute}{06/2021 - 07/2021}
\begin{details}
    \item Developed a computer vision tool that converts images to line art for a robotic drawing application.
\end{details}

\vspace{-2mm}

\sectiontitle{Publications}

\vspace{-5mm}

\renewcommand{\refname}{}
\renewcommand{\refname}{\vspace{-\baselineskip}}

\bibliographystyle{unsrt}
\nocite{*}
\bibliography{pub}

\vspace{-3mm}

\sectiontitle{Research \& Robotics Projects}

\entrytitle{Bootstrapped DQN Scaling Laws}{JAX, Deep Reinforcement Learning}{2025}
\begin{details}
    \item Conducted a large-scale empirical study discovering a unified scaling law for ensemble-based exploration in RL, using a high-performance, multi-GPU JAX framework for the research.
\end{details}

\entrytitle{Lyapunov-Stabilized Truncated Backpropagation}{Python, JAX, Deep Reinforcement Learning}{2024}
\begin{details}
    \item Implemented a novel RL framework using Lyapunov stability factors to stabilize training in long-horizon continuous control and locomotion tasks (Brax).
\end{details}

\entrytitle{Stack-Associated Beam Tracing}{C++20, Computational Geometry}{2022}
\begin{details}
    \item Developed a 3D rendering engine in C++ from scratch, implementing sparse voxel octree traversal and GJK collision detection for efficient scene queries, foundational for robotics simulation.
\end{details}

\entrytitle{IEFT-PFN for Hyperparameter Optimization}{JAX, Equinox, Transformers, AutoML}{2025}
\begin{details}
    \item Built a Prior-Data Fitted Network (PFN) using a causal transformer for hyperparameter optimization, demonstrating expertise in meta-learning and transformer architectures.
\end{details}

\entrytitle{RoboCup Junior OnStage Performance}{Robotics, C++, Computer Vision}{2019}
\begin{details}
    \item Designed and built autonomous wheeled robots for a theatrical performance, developing the vision and control software stack. Awarded Best Software Solution at the World Championships.
\end{details}

\vspace{-2mm}

\sectiontitle{Honours and Awards}
\begin{itemize}
    \item \textbf{Best Software Solution}, RoboCup World Championships, Sydney (2019).
    \item \textbf{1st Place}, RoboCup Junior National Competitions (2017, 2018, 2019).
    \item \textbf{1st Place}, Bunq Hackathon 6 (2025).
    \item \textbf{2nd Place \& Special Prize}, Epoch AI Hackathon (2024).
    \item \textbf{Silver Medal}, AIIJC International AI Competition for Juniors for a sign language recognition app.
\end{itemize}

\vspace{-2mm}

\sectiontitle{Open Source Contributions}
\begin{itemize}
    \item Contributed to the JAX ecosystem (\href{https://github.com/patrick-kidger/equinox}{Equinox}, \href{https://github.com/google/jaxtyping}{jaxtyping}), RL environments (\href{https://github.com/RobertTLange/gymnax}{Gymnax}), and a C++ collision detection library (libccd), fixing critical bugs and adding new features.
\end{itemize}

\end{document}
