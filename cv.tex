\documentclass[11pt, a4paper]{article}
\usepackage[utf8]{inputenc}
\usepackage{geometry}
\usepackage{enumitem}
\usepackage{hyperref}
\usepackage{xcolor}
\usepackage{fontawesome5} % For icons like GitHub/LinkedIn

\geometry{left=2.5cm, right=2.5cm, top=1.5cm, bottom=1.5cm}
\pagenumbering{gobble}
\definecolor{primaryColor}{RGB}{0, 102, 204}

\setlength{\parindent}{0pt}
\setlength{\parskip}{1.5ex} % This controls the space BETWEEN entries. Adjust as needed.

% Command for a standard entry title line (left-aligned title, right-aligned date)
\newcommand{\entrytitle}[3]{%
    \noindent\textbf{#1}, #2\hfill#3%
}

% A custom list for details within an entry (e.g., GPA, bullet points)
% This is very compact and sits tightly under the title.
\newlist{details}{itemize}{1}
\setlist[details]{
    label=\textbullet,
    leftmargin=1.5em,
    topsep=0.2ex,
    partopsep=-0pt,
    itemsep=0.2ex,
    parsep=0pt
}

\hypersetup{
    colorlinks=true,
    linkcolor=primaryColor,
    filecolor=primaryColor,
    urlcolor=primaryColor,
    citecolor=primaryColor,
}

\setlist[itemize]{noitemsep, topsep=0pt, leftmargin=*}

\newcommand{\sectiontitle}[1]{\vspace{3mm}{\color{primaryColor}\noindent\textbf{#1}}\vspace{1.5mm}\hrule\vspace{2mm}}
\newcommand{\experience}[4]{\textbf{#1}, #2 \hfill \vspace{-4mm}#3\\#4\vspace{3mm}}

\begin{document}

\begin{center}
    \LARGE{\textbf{\color{primaryColor}Roman Knyazhitskiy}}\\
    \vspace{2mm}
    \normalsize{
        \faEnvelope~\href{mailto:cv@knyaz.tech}{cv@knyaz.tech} \quad|\quad
        \faGlobe~\href{https://knyaz.tech}{knyaz.tech} \quad|\quad
        \faGithub~\href{https://github.com/knyazer}{knyazer}
    }
\end{center}

\sectiontitle{Summary}
\noindent
Cambridge MPhil student combining deep technical expertise in JAX/PyTorch with real-world impact. Built differentiable simulators enabling more efficient RL algorithms, developed MPC systems improving robot trajectory accuracy, and contributed to major open-source ML libraries. Proven track record spans 10 years of robotics experience, from hackathon victories to published conference papers.

\vspace{-2mm}


\sectiontitle{Education}

\entrytitle{MPhil in MLMI (Machine Learning)}{University of Cambridge}{10/2025 - 09/2026}

\entrytitle{BSc Computer Science and Engineering}{TU Delft}{09/2022 - 07/2025}
\begin{details}
    \item GPA: 8.7/10. Graduated with Honours.
\end{details}

\vspace{-2mm}


\sectiontitle{Work Experience}

\entrytitle{Research Associate}{TU Delft}{03/2023 - 08/2025}
\begin{details}
    \item Maintained regular research engagement through weekly supervisory consultations.
    \item Participated in departmental research meetings and academic discussions.
    \item Advanced research competencies through active involvement in university research programs.
\end{details}

\entrytitle{Head of AI}{Delft Mercurians (Student Robotics Team)}{05/2023 - Present (Part-Time)}
\begin{details}
    \item Led a team of 2-5 engineers in developing AI control systems for RoboCup competitions.
    \item Designed and implemented a Model Predictive Control (MPC) system for robot trajectory optimization, significantly enhancing path-following accuracy and tactical execution.
    \item Introduced a continuous-time, differentiable simulator in JAX, providing the foundation for the implementation of more efficient learning algorithms.
    \item Ensured AI solution robustness via CI/CD, runtime type checking, and a comprehensive test suite.
\end{details}

\entrytitle{Software Engineer}{Delft Mercurians (Student Robotics Team)}{05/2023 - 10/2023}
\begin{details}
    \item Developed a sensor fusion toolkit for our wheeled platform, leading to more accurate Kalman Filter calibration.
    \item Analyzed past movement data to establish sensor fusion performance benchmarks and improve system reliability.
\end{details}

\vspace{-2mm}

\sectiontitle{Honours and Awards}
\begin{itemize}
    \item \textbf{1st Place}, Bunq Hackathon 6 (2025), team of 4 against 50+ teams with a prize of €30,000.
    \item \textbf{2nd Place \& Special Prize}, Epoch AI Hackathon (2024).
    \item Recruited by San Francisco robotics startup for founding engineer role; prioritized continued education at Cambridge.
    \item \textbf{'Best Software Solution'} award, RoboCup World Championships, Sydney (2019).
    \item \textbf{1st Place}, RoboCup Junior National Competitions (2017, 2018, 2019).
\end{itemize}


\vspace{-2mm}

\sectiontitle{Open Source Contributions}
\begin{itemize}
    \item Enhanced functionalities in \href{https://github.com/google/jaxtyping}{jaxtyping} and \href{https://github.com/patrick-kidger/equinox}{Equinox}, resolving multiple issues and enabling IPython runtime type checking.
    \item Contributed to \href{https://github.com/RobertTLange/gymnax}{Gymnax}, a widely used JAX RL environments collection with 800+ stars on GitHub.
\end{itemize}


\vspace{-2mm}

\sectiontitle{Publications}
\noindent
\textbf{Knyazhitskiy, R.}, Van der Vaart, P.R. "A Simple Scaling Model for Bootstrapped DQN". Submitted to AAAI Conference on Artificial Intelligence (2025). Under review.

\noindent
Luijmes, J., Gielisse, A., \textbf{Knyazhitskiy, R.}, van Gemert, J. (2025). "ARC: Anchored Representation Clouds for High-Resolution INR Classification". \textit{Accepted at ICLR 2025 Workshop on Weight Space Learning}.

\vspace{-2mm}

\sectiontitle{Selected Projects}

\entrytitle{Lyapunov Discounting for BPTT Optimization}{JAX, Differentiable Simulation}{2024}
\begin{details}
    \item Novel stability improvement method for backpropagation through differentiable simulators, achieving significant gains over windowed BPTT, 5-fold increase in the peak performance achieved on Brax environments
\end{details}

\entrytitle{pytest-mut: High-Performance Mutation Testing}{Python, Parallel Computing}{2024}
\begin{details}
    \item Developed mutation testing library achieving 10-15x speedup over alternatives through novel partial parallelization strategies
\end{details}

\entrytitle{High-Frequency Weather Forecasting}{Hierarchical Bayesian Models, CNF}{2023}
\begin{details}
    \item Developed hybrid forecasting system combining continuous normalizing flows with autoregressive models for improved temporal resolution
\end{details}

\entrytitle{Quadrotor Control System}{C/C++, Embedded Systems}{2023}
\begin{details}
    \item Implemented low-level control algorithms with hardware integration for autonomous quadrotor flight control
\end{details}

\vspace{-2mm}

\sectiontitle{Skills}
\begin{itemize}
    \item \textbf{Machine Learning:} JAX, Equinox, PyTorch, OpenCV
    \item \textbf{Programming:} Python, Rust, C/C++, Full stack web development
    \item \textbf{Robotics \& Hardware:} CAD (Autodesk Inventor/SolidWorks), embedded systems programming, sensor integration
    \item \textbf{Tools:} Git, CI/CD, containerization
\end{itemize}

\end{document}
