\documentclass[11pt, a4paper]{article}
\usepackage[utf8]{inputenc}
\usepackage{geometry}
\usepackage{enumitem}
\usepackage{hyperref}
\usepackage{xcolor}
\usepackage{fontawesome5}

\geometry{left=2.5cm, right=2.5cm, top=1.5cm, bottom=1.5cm}
\pagenumbering{gobble}
\definecolor{primaryColor}{RGB}{0, 102, 204}

\setlength{\parindent}{0pt}
\setlength{\parskip}{1.5ex}

\newcommand{\entrytitle}[3]{
    \noindent\textbf{#1}, #2\hfill#3
}

\newlist{details}{itemize}{1}
\setlist[details]{
    label=\textbullet,
    leftmargin=1.5em,
    topsep=0.2ex,
    partopsep=-0pt,
    itemsep=0.2ex,
    parsep=0pt
}

\hypersetup{
    colorlinks=true,
    linkcolor=primaryColor,
    filecolor=primaryColor,
    urlcolor=primaryColor,
    citecolor=primaryColor,
}

\setlist[itemize]{noitemsep, topsep=0pt, leftmargin=*}

\newcommand{\sectiontitle}[1]{\vspace{3mm}{\color{primaryColor}\noindent\textbf{#1}}\vspace{1.5mm}\hrule\vspace{2mm}}
\newcommand{\experience}[4]{\textbf{#1}, #2 \hfill \vspace{-4mm}#3\\#4\vspace{3mm}}

\begin{document}

\begin{center}
    \LARGE{\textbf{\color{primaryColor}Roman Knyazhitskiy}}\\
    \vspace{2mm}
    \normalsize{
        \faEnvelope~\href{mailto:cv@knyaz.tech}{cv@knyaz.tech} \quad|\quad
        \faGlobe~\href{https://knyaz.tech}{knyaz.tech} \quad|\quad
        \faGithub~\href{https://github.com/knyazer}{knyazer}
    }
\end{center}

\sectiontitle{Summary}
\noindent
Master's student with a passion for Machine Learning and Robotics. Experienced in prototyping applications using LLMs, computer vision, and implicit neural representations. Skilled in translating research concepts into tangible, user-facing prototypes, demonstrated through projects like a real-time sign language interpreter and an LLM-powered literature search engine.

\vspace{-2mm}


\sectiontitle{Education}

\entrytitle{MPhil in MLMI (Machine Learning)}{University of Cambridge}{10/2025 - 09/2026}

\entrytitle{BSc Computer Science and Engineering}{TU Delft}{09/2022 - 07/2025}
\begin{details}
    \item GPA: 8.7/10. Distinction (cum laude, top 5\%) + Honours.
\end{details}

\vspace{-2mm}


\sectiontitle{Work Experience}

\entrytitle{Machine Learning Engineer}{Delft Mercurians}{05/2023 - 09/2025}
\begin{details}
    \item Started as an individual contributor, then grew a team of ML control up to 5 people.
    \item Developed a PyQt interface for Kalman filter calibration, enabling interactive data visualization.
    \item Integrated Python-based ML models into a performance-critical Rust codebase.
    \item Built and maintained a JAX-based differentiable physics simulator for training control models.
\end{details}

\entrytitle{Research Associate}{TU Delft}{03/2023 - 08/2025}
\begin{details}
    \item Researched applications of Large Language Models in software engineering, focusing on code generation.
    \item Investigated transformer-based Prior-Data Fitted Networks (PFNs) and participated in research discussions on diffusion models for generative tasks.
\end{details}

\entrytitle{Applied Machine Learning Intern}{Central Robotics Institute}{06/2021 - 07/2021}
\begin{details}
    \item Developed a computer vision tool to convert images into line art for a robotic drawing application.
\end{details}

\vspace{-2mm}

\sectiontitle{Selected Projects}

\entrytitle{Silver-qt: Sign Language Recognition System}{Python, PyQt, Computer Vision, ONNX}{2019}
\begin{details}
    \item Developed a consumer-facing application for real-time sign language recognition from video streams.
    \item Built a cross-platform GUI with PyQt5/QML and used ONNX Runtime for optimized ML model inference.
    \item Implemented a multi-stage vision pipeline: YOLOv5 for hand detection, Autoencoder for feature extraction, and an LSTM for sequence classification.
\end{details}

\entrytitle{Spectral: Speech Analysis Web Platform}{Python, FastAPI, SvelteKit, Docker}{2024}
\begin{details}
	\item Built and deployed a full-stack, containerized web platform for atypical speech analysis.
	\item Designed a microservices architecture with a Python/FastAPI backend and a SvelteKit frontend, enabling researchers to upload and analyze audio data through a user-friendly interface.
\end{details}

\entrytitle{Better Literature Search Project}{LLMs, Embedding Models, Python}{2024}
\begin{details}
    \item Co-founded a startup building a user-facing literature search engine similar to Elicit.
    \item Designed and prototyped a system using embedding models for semantic filtering and an LLM for final paper scoring and ranking, demonstrating a passion for building intelligent products.
\end{details}

\vspace{-2mm}

\sectiontitle{Publications}

\vspace{-5mm}

\renewcommand{\refname}{}
\renewcommand{\refname}{\vspace{-\baselineskip}}

\bibliographystyle{unsrt}
\nocite{*}
\bibliography{pub} 

\vspace{-3mm}

\sectiontitle{Honours and Awards}
\begin{itemize}
    \item \textbf{1st Place}, Bunq Hackathon 6 (2025).
    \item \textbf{2nd Place \& Special Prize}, Epoch AI Hackathon (2024).
    \item \textbf{Silver Medal}, AIIJC International AI Competition for Juniors for Sign language recognition app.
    \item \textbf{Best Software Solution}, RoboCup World Championships, Sydney (2019).
\end{itemize}


\vspace{-2mm}

\sectiontitle{Open Source Contributions}
\begin{itemize}
    \item Enhanced functionalities in Google's \href{https://github.com/google/jaxtyping}{jaxtyping} and \href{https://github.com/patrick-kidger/equinox}{Equinox}, resolving multiple issues.
    \item Contributed to \href{https://github.com/RobertTLange/gymnax}{Gymnax}, a widely used JAX RL environments collection.
    \item Improved libccd (C++ collision detection library) by fixing a critical infinite loop bug.
\end{itemize}

\end{document}
