\documentclass[11pt, a4paper]{article}
\usepackage[utf8]{inputenc}
\usepackage{geometry}
\usepackage{enumitem}
\usepackage{hyperref}
\usepackage{xcolor}
\usepackage{fontawesome5} % For icons like GitHub/LinkedIn

\geometry{left=2.5cm, right=2.5cm, top=1.5cm, bottom=1.5cm}
\pagenumbering{gobble}
\definecolor{primaryColor}{RGB}{0, 102, 204}

\setlength{\parindent}{0pt}
\setlength{\parskip}{1.5ex} % This controls the space BETWEEN entries. Adjust as needed.

% Command for a standard entry title line (left-aligned title, right-aligned date)
\newcommand{\entrytitle}[3]{%
    \noindent\textbf{#1}, #2\hfill#3%
}

% A custom list for details within an entry (e.g., GPA, bullet points)
% This is very compact and sits tightly under the title.
\newlist{details}{itemize}{1}
\setlist[details]{
    label=\textbullet,
    leftmargin=1.5em,
    topsep=0.2ex,
    partopsep=-0pt,
    itemsep=0.2ex,
    parsep=0pt
}

\hypersetup{
    colorlinks=true,
    linkcolor=primaryColor,
    filecolor=primaryColor,
    urlcolor=primaryColor,
    citecolor=primaryColor,
}

\setlist[itemize]{noitemsep, topsep=0pt, leftmargin=*}

\newcommand{\sectiontitle}[1]{\vspace{3mm}{\color{primaryColor}\noindent\textbf{#1}}\vspace{1.5mm}\hrule\vspace{2mm}}
\newcommand{\experience}[4]{\textbf{#1}, #2 \hfill \vspace{-4mm}#3\\#4\vspace{3mm}}

\begin{document}

\begin{center}
    \LARGE{\textbf{\color{primaryColor}Roman Knyazhitskiy}}\\
    \vspace{2mm}
    \normalsize{
        \faEnvelope~\href{mailto:cv@knyaz.tech}{cv@knyaz.tech} \quad|\quad
        \faGlobe~\href{https://knyaz.tech}{knyaz.tech} \quad|\quad
        \faGithub~\href{https://github.com/knyazer}{knyazer}
    }
\end{center}

\sectiontitle{Summary}
\noindent
Incoming MPhil student at the University of Cambridge with a track record in AI research and software engineering. My experience spans machine learning, computer vision, and robotics. I am passionate about developing innovative, high-impact solutions and aim to contribute to cutting-edge research and development teams.

\vspace{-2mm}


\sectiontitle{Education}

\entrytitle{MPhil in MLMI (Machine Learning)}{University of Cambridge}{10/2025 - 09/2026}

\entrytitle{BSc Computer Science and Engineering}{TU Delft}{09/2022 - 07/2025}
\begin{details}
    \item GPA: 9/10. Graduated with Honours.
\end{details}

\vspace{-2mm}


\sectiontitle{Work Experience}

\entrytitle{Head of AI}{Delft Mercurians (Student Robotics Team)}{05/2023 - Present (Part-Time)}
\begin{details}
    \item Led a team of 3 engineers in developing AI control systems for RoboCup competitions.
    \item Designed and implemented a Model Predictive Control (MPC) system for robot trajectory optimization, significantly enhancing path-following accuracy and tactical execution.
    \item Spearheaded the creation of a continuous-time, differentiable simulator in JAX, providing the foundation for the implementation of more efficient learning algorithms.
    \item Ensured AI solution robustness via CI/CD, runtime type checking, and a comprehensive test suite.
\end{details}

\entrytitle{Software Engineer}{Delft Mercurians (Student Robotics Team)}{05/2023 - 10/2023}
\begin{details}
    \item Developed a sensor fusion toolkit for our wheeled platform, leading to more accurate Kalman Filter calibration.
    \item Analyzed past movement data to establish sensor fusion performance benchmarks and improve system reliability.
\end{details}

\vspace{-2mm}

\sectiontitle{Honours and Awards}
\begin{itemize}
    \item \textbf{1st Place}, Bunq Hackathon 6 (2025), against 50+ teams with a prize of €30,000.
    \item \textbf{2nd Place \& Special Prize}, Epoch AI Hackathon (2024).
    \item \textbf{'Best Software Solution'} award, RoboCup World Championships, Sydney (2019).
    \item \textbf{1st Place}, RoboCup Junior National Competitions (2017, 2018, 2019).
\end{itemize}


\vspace{-2mm}

\sectiontitle{Open Source Contributions}
\begin{itemize}
    \item Enhanced functionalities in \href{https://github.com/google/jaxtyping}{jaxtyping} and \href{https://github.com/patrick-kidger/equinox}{Equinox}, resolving multiple issues and enabling IPython runtime type checking.
    \item Currently (as of 2025) a semi-official maintainer of Gymnax \href{https://github.com/RobertTLange/gymnax} - a widely used JAX RL environments collection, with 800 stars on GitHub.
\end{itemize}


\vspace{-2mm}

\sectiontitle{Publications}
\noindent
Luijmes, J., Gielisse, A., \textbf{Knyazhitskiy, R.}, van Gemert, J. (2025). "ARC: Anchored Representation Clouds for High-Resolution INR Classification". \textit{arXiv preprint arXiv:2503.15156}.

\begin{itemize}
    \item Accepted for presentation at the ICLR 2025 Workshop on Weight Space Learning.
\end{itemize}

\vspace{-2mm}

\sectiontitle{Skills}
\begin{itemize}
    \item \textbf{Machine Learning:} PyTorch, JAX, Equinox, HuggingFace, OpenCV, scikit-learn, W\&B
    \item \textbf{Programming:} Python, Rust, C/C++, Java, SQL, HTML/CSS/JS
    \item \textbf{DevOps \& Tools:} Git, GitHub Actions (CI/CD), Docker, JIRA, Ruff, Flake8
    \item \textbf{Languages:} English, Russian
\end{itemize}

\end{document}
