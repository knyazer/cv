\documentclass[11pt, a4paper]{article}
\usepackage[utf8]{inputenc}
\usepackage{geometry}
\usepackage{enumitem}
\usepackage{hyperref}
\usepackage{xcolor}
\usepackage{fontawesome5}

\geometry{left=2.5cm, right=2.5cm, top=1.5cm, bottom=1.5cm}
\pagenumbering{gobble}
\definecolor{primaryColor}{RGB}{0, 102, 204}

\setlength{\parindent}{0pt}
\setlength{\parskip}{1.5ex}

\newcommand{\entrytitle}[3]{
    \noindent\textbf{#1}, #2\hfill#3
}

\newlist{details}{itemize}{1}
\setlist[details]{
    label=\textbullet,
    leftmargin=1.5em,
    topsep=0.2ex,
    partopsep=-0pt,
    itemsep=0.2ex,
    parsep=0pt
}

\hypersetup{
    colorlinks=true,
    linkcolor=primaryColor,
    filecolor=primaryColor,
    urlcolor=primaryColor,
    citecolor=primaryColor,
}

\setlist[itemize]{noitemsep, topsep=0pt, leftmargin=*}

\newcommand{\sectiontitle}[1]{\vspace{3mm}{\color{primaryColor}\noindent\textbf{#1}}\vspace{1.5mm}\hrule\vspace{2mm}}
\newcommand{\experience}[4]{\textbf{#1}, #2 \hfill \vspace{-4mm}#3\\#4\vspace{3mm}}

\begin{document}

\begin{center}
    \LARGE{\textbf{\color{primaryColor}Roman Knyazhitskiy}}\\
    \vspace{2mm}
    \normalsize{
        \faEnvelope~\href{mailto:cv@knyaz.tech}{cv@knyaz.tech} \quad|\quad
        \faGlobe~\href{https://knyaz.tech}{knyaz.tech} \quad|\quad
        \faGithub~\href{https://github.com/knyazer}{knyazer}
    }
\end{center}

\sectiontitle{Summary}
\noindent
Machine learning engineer with a strong background in robotics, deep reinforcement learning, and large language models. Extensive hands-on experience implementing and training autoregressive transformer architectures from scratch in JAX. Proven ability in building AI control systems for robotic platforms, conducting large-scale training experiments, and contributing to performance-critical C++ codebases.

\vspace{-2mm}


\sectiontitle{Education}

\entrytitle{MPhil in MLMI (Machine Learning)}{University of Cambridge}{10/2025 - 07/2026}

\entrytitle{BSc Computer Science and Engineering}{TU Delft}{09/2022 - 07/2025}
\begin{details}
    \item GPA: 8.7/10. Distinction (Cum Laude, top 5\%) + Honours.
\end{details}

\vspace{-2mm}


\sectiontitle{Work Experience}

\entrytitle{Machine Learning Engineer}{Delft Mercurians}{05/2023 - 09/2025}
\begin{details}
    \item Led the development of AI control systems for a competitive robotics team (RoboCup).
    \item Designed and built a continuous-time differentiable physics simulator in JAX for training control policies.
    \item Integrated Python-based ML models with the main Rust codebase for real-time operation.
    \item Developed and implemented a Model Predictive Control (MPC) system for real-time trajectory optimization.
\end{details}

\entrytitle{Research Associate}{TU Delft}{03/2023 - 08/2025}
\begin{details}
    \item Researched applications of transformer-based models (Prior-Data Fitted Networks) for meta-learning.
    \item Investigated the use of Large Language Models for software engineering tasks like code generation.
\end{details}

\entrytitle{Applied Machine Learning Intern}{Central Robotics Institute}{06/2021 - 07/2021}
\begin{details}
    \item Developed computer vision algorithms for a robotic drawing application, including image segmentation and path optimization.
\end{details}

\vspace{-2mm}

\sectiontitle{Selected Projects}

\entrytitle{Nano JAX GPT: Scalable Transformer Implementation}{JAX, Equinox, Deep Learning}{2023}
\begin{details}
    \item Implemented a GPT-style autoregressive transformer model from scratch in JAX/Equinox, featuring multi-head causal self-attention.
    \item Architected for distributed training across multiple accelerators with mixed-precision (bfloat16).
    \item Researched and implemented inference optimizations including speculative decoding and multi-token prediction.
\end{details}

\entrytitle{Bootstrapped DQN Scaling Laws}{JAX, Deep RL Research, HPC}{2025}
\begin{details}
    \item Conducted a large-scale reinforcement learning study (>40,000 experiments) finding a novel scaling law for ensemble-based exploration methods.
    \item Designed and implemented a high-performance JAX pipeline for multi-GPU distributed training and automated statistical analysis.
\end{details}

\entrytitle{Stack-Associated Beam Tracing}{C++20, Graphics Rendering}{2022}
\begin{details}
    \item Implemented a 3D rendering engine in C++20 based on sparse voxel octree traversal for efficient ray queries.
    \item Developed a custom linear algebra library, geometric algorithms for collision detection (GJK), and an interactive rendering pipeline.
\end{details}

\entrytitle{Silver-qt: Sign Language Recognition System}{Python, Computer Vision, ONNX}{2019}
\begin{details}
    \item Developed a real-time vision system using a deep learning pipeline (YOLOv5, Autoencoder, LSTM).
    \item Built a cross-platform GUI and used ONNX Runtime for optimized multi-threaded inference.
\end{details}

\vspace{-2mm}

\sectiontitle{Publications}

\vspace{-5mm}

\renewcommand{\refname}{}
\renewcommand{\refname}{\vspace{-\baselineskip}}

\bibliographystyle{unsrt}
\nocite{*}
\bibliography{pub}

\vspace{-3mm}

\sectiontitle{Honours and Awards}
\begin{itemize}
    \item \textbf{1st Place}, Bunq Hackathon 6 (2025).
    \item \textbf{2nd Place \& Special Prize}, Epoch AI Hackathon (2024).
    \item \textbf{Best Software Solution}, RoboCup World Championships, Sydney (2019).
    \item \textbf{1st Place}, RoboCup Junior National Competitions (2017, 2018, 2019).
    \item \textbf{Silver Medal}, AIIJC International AI Competition for Juniors -- Sign language recognition application.
\end{itemize}

\vspace{-2mm}

\sectiontitle{Open Source Contributions}
\begin{itemize}
    \item Enhanced functionalities in \href{https://github.com/google/jaxtyping}{jaxtyping} and \href{https://github.com/patrick-kidger/equinox}{Equinox}, resolving multiple issues and enabling IPython runtime type checking.
    \item Contributed to \href{https://github.com/RobertTLange/gymnax}{Gymnax}, a widely used JAX RL environments collection.
    \item Improved \textbf{libccd (collision detection library in C++)} fixing a critical corner-case causing an infinite loop.
\end{itemize}

\end{document}
